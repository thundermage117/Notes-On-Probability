\documentclass{article}

\begin{document}
\section{Probability Space}
There are often various approaches to probability each with its own advantages
and disadvantages.

Experiment$\rightarrow$ procedure that can be infinitely repeated and has a
well-defined set of possible outcomes, known as the sample space.

The observation/result of the experiment are termed as outcomes.

\subsection{Classical Approach}
        Probability of an event $E$ is defined to be: $$P(E)=\frac{Number\; of\;
         outcomes\;in\; E}{Total\; number\; of\; outcomes}$$
         Some examples are tossing a coin or rolling a die.
         Disadvantages:
         \begin{itemize}
             \item Unable to model biases. It says nothing about cases where no
             physical symmetry exists.
             \item Doesn't deal with cases where total outcomes are infinite.
         \end{itemize}

         \subsection{Frequentist Approach}
         Also known as the relative frequency approach or frequentism. It defines
         an event's probability as the limit of its relative frequency in many
         trials.

         Probability is defined to be:
         $$ P(E)=\lim_{n \to \infty} \frac{n_E}{n}$$
         where an experiment is conducted $n$ times and event $E$ occurs $n_E$
         times.
         Disadvantages:
         \begin{itemize}
             \item It isn't efficient to conduct an experiment multiple times
             just to find the probability of an event occuring.
             \item It is unable to deal with subjective belief. Eg: Suppose a
             cricket expert says there is a $50\%$ of RCB winning the IPL this
             year. It doesn't mean that the RCB has won half the titles in the
             past.
         \end{itemize}
%
         \subsection{Axiomatic Approach}

         \subsubsection{Probability Space}

The triple ($S$, $F$, $P$) is referred to as a probability space where:
\begin{itemize}
    \item $S$ : Sample space, set of all possible outcomes of the experiment.
    \item $F$ : Event Space, collection of events
    \item $P$ : Probability Measure
\end{itemize}

$S$ can either be finite or countably infinite or uncountably infinite.








\end{document}
